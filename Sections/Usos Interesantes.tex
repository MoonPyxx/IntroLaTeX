\section{Casos de uso interesantes y aprovechar al 100\% \LaTeX}

\begin{frame}{Automatización}
  \begin{itemize}
    \item Con \alert{\LaTeX{}} puedes automatizar gran parte de la creación de un documento ya que no requiere que modifiques el formato en absoluto, se va adaptando a tu configuración y lo que escribes es flexible para todas las plantillas.
    \item Con \textbf{pdflatex} existe la opción de crear scripts para terminal que vayan compilando tu proyecto automáticamente y generando varias copias.
  \end{itemize}
\end{frame}

\begin{frame}[fragile]{Colaboración}
  \begin{itemize}
    \item \textbf{Overleaf}
        Overleaf te permite colaborar con otras personas fácilmente, aunque desde la versión gratuita no es tan fácil.
    \item \textbf{Git}
        Debido a que es texto plano, LaTeX funciona muy bien con Git y permite la sincronización y colaboración de manera muy fácil.
  \end{itemize}
  Además, gracias a que se puede editar desde cualquier editor, no vas a tener problemas de compatbilidad al enviarlo a alguien, no ocurre como con Word/OpenOffice, etc.
\end{frame}
\begin{frame}[fragile]{Soporte}
  \begin{itemize}
    \item \textbf{Google}
    \item \textbf{ChatGPT/Otros}
    \item \textbf{TeX FAQ} \url{https://texfaq.org}
    \item \textbf{LaTeX Stack Exchange:} \url{https://tex.stackexchange.com}
  \end{itemize}
\end{frame}
\begin{frame}[fragile]{Casos de uso interesantes}
  \begin{itemize}
    \item Crea tu propia página web transformando tus archivos .tex en .html con \textbf{TeX4ht}: \url{https://tug.org/tex4ht/}
    \item O tu CV con \textbf{Awesome-CV}: \url{https://github.com/posquit0/Awesome-CV}
    \item Descubre temas para tus presentaciones (beamer): \url{https://hartwork.org/beamer-theme-matrix/}
    \item Miles de plantillas para todo tipo de usos: \url{https://www.latextemplates.com/}
    \item Integración con Markdown, \textbf{Obsidian}
  \end{itemize}
\end{frame}
