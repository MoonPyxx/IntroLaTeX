\section{Cómo usar \LaTeX}

\begin{frame}{Introducción. ¿Qué es LaTeX?}
  \begin{itemize}
    \item \alert{\LaTeX{}} es una \textbf{herramienta} sofisticada para profesionalizar tus documentos. NO es un editor de textos, no tiene una interfaz gráfica en sí.
    \item \textbf{Tú escribes} y la herramienta se encarga de la presentación.
    \item Se puede usar para todo tipo de usos, desde artículos a presentaciones como esta e incluso gráficos y posters.
    \item \textbf{\textcolor{green}{Ventajas}:}
          \begin{itemize}
            \item Separa el formato del contenido.
            \item Alta capacidad para estructurar información.
            \item Gran integración con sistemas como git.
            \item Evita muchos de los problemas de los procesadores de texto convencionales. 
            \item Muy personalizable gracias a los paquetes.
          \end{itemize}
    \item \textbf{\textcolor{red}{Desventajas}:}
          \begin{itemize}
            \item No incluye una interfaz gráfica, hay que recurrir a programas externos.
            \item No es demasiado útil para documentos no estructurados.
          \end{itemize}
  \end{itemize}
\end{frame}

\begin{frame}[fragile]{Instalación}
  \begin{itemize}
    \item \textbf{Windows/Linux}
          \begin{itemize}
            \item TeXLive \url{https://www.tug.org/texlive/}
            \item TeXStudio \url{https://www.texstudio.org/}
            \item pdflatex, chktex (terminal)
          \end{itemize}
    \item \textbf{Mac}
          \begin{itemize}
            \item MacTeX \url{http://www.tug.org/mactex/}
          \end{itemize}
  \end{itemize}
  Todo el código en \LaTeX es simple texto, se puede abrir con cualquier editor. Por tanto, puedes usar:
  \begin{itemize}
    \item{Visual Studio Code}
    \item {Vim}
    \item {Emacs}
    \item {Overleaf} \url{https://www.overleaf.com}
  \end{itemize}
\end{frame}
